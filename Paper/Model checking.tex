\subsection{Model checking}
As is known to all, when a model had been established, there must be a checkout about the model. As we all know, when a people play this kind of game, his steps to solve the problem may like this: first, reading rules, and then, according to the rules to find out the easiest location which we called it AR (absolute rules), after that we can use the relationship between the other rules and the ARs to find the certain location.

As for level 33, the steps of people’s solving maybe like this: first, reading rules, we can determine the approximate location of the pandas, the lions and the tigers, then we read other rules we can find the final location of the lions, pandas, tigers and cat. Then the location of the turtle can be find out. 

We have compiled the model we created with the program. We entered the 33 rules of this game into our program. We see the results as fig.\ref{fig:result}.
\begin{figure}[h]
	\centering 
	\includegraphics[width=0.7\linewidth]{figures/result} 
	\caption{result.}
	\label{fig:result}   
\end{figure}

Therefore, our model is consistent with the final result. In general, our model can be used to solve and generalize this game.
