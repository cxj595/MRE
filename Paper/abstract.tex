\begin{abstract}
\renewcommand{\baselinestretch}{1.0}
Gaming is the nature of human being, the source of creativity, and the drive to develope childs' thinking power. Thus how to design a proper puzzle providing both flexibility and challenge is a significant issue for game disigners. Especially in this fast-paced digital age, how to bring computer-aided workflow to puzzle design is having an increasingly importance.
Targeted on this "Wonder Island" Game, we invent a Model driven by Rule Engine ("MRE") to automaticly evaluate difficulty and solving those kind of generalized Sudoku games, which implement information theory and complexity thory to analyze the gaming status and efficiently. Then we've run some test cases bringing by our own to verify its high performance, including evaluaing and solving more complex games having more grids.
To bring it to the next level, we modify the MRE to suit more flexible rubrics, even more dimensions and invent a brand new puzzle game " " that is similar to Sudoku, but working in 3 dimensions. It's suppoed to develope childs' both logical inferance and spatial imagination. Moreover, we've done a literal groundbreaking job to implement MRE reversly to aid puzzle designing process and even design new game levels on its own. After Turing-like tests, they was proved similar to thus designed by human designers, of which is described in detail in the non-technical report.

In gerneral, the main contribution of our original work is:

Implementing a set of concepts in informaion theory to promote the MRE choosing logic, incluing self-information, entropy, and conditional entropy.
Inventing the concept of "thinking energy" to link the solving difficulty and computing complxity, then using the latter one (with some modification to simulate and represent how real brain works) to help modeling the difficulty of puzzles.

Inspired by the solution of Sudoku, applying the Brute-force and backtracing algorithm powered with our puzzle-oriented pruning strategy to solve the puzzle automaticly, meanwhile accumulating the "thinking power" to evaluate the difficulties.

Introducing the term "Absolute Rule" and "Relative Rule" to divide rules in 2 parts by its property, and inventing an engine (the Rule Engine) to tranform and simplify the rules repeatedly, which could prune in high efficience.
Using the MRE both in the normal way to evaluate and reverse way to create new game levels.


\end{abstract}

%\begin{keywords}
% 写关键词
%***
%\end{keywords}
