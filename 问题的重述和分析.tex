\newpage
\renewcommand{\baselinestretch}{1.5}
\section{Restatements}
There are kinds of games in our life. Through the game, we can learn a lot. When we disign a game, we also need a lot of knowledge from varieties of subjects.  There is a kind of game for children whose age is over 6 which is called "Wonder Island", which is aimed at training the sense of permutation and combination, spatial position and neighboring relations, logic reasoning, etc.  There are 9 kinds of animals which include an elephant , a monkey, a pig, a tiger, a cat, a turtle, two hippos, 3 lions, and 5 pandas.  There are also some fruits in this game, 3 fruits( Apple, Mango, Orange)  represent the vertical axis another three fruits (Pineapple, Banana and Strawberry) represent horizontal axis. For each level, there are different rules for some of 9 animals and 6 fruits. Through these rules, the children should find out the location of each animal on this " Wonder Island"  which is a kind of 3 x 3 squares. And there are three problems we need to solve.

\begin{enumerate}[(1)]
\item The difficulty of this game is gradually increasing with the level. What needing to do is to design a mathematical model to measure the difficulty of each level of this game, and evaluate the rationality of the model.

\item In each level, there are some restrictions which can be used to determining the location of each animal. Now what should be done is to establish a model to find out the location of each animal by using those restrictions. Meanwhile, it also require to determine if this solution is unique.

\item There should be more games like "Wonder Island", because they can use a samilar model. This problem ask to design a samilar model like "Wonder Island" based on the models in problem1 and 2.
\end{enumerate}
\section{Problem Anslysis}
The principles of this game is similar to the principle of Sudoku and Rubik's Cube, so we can use some principles which similar to Sudoku and Rubik's Cube to solve the problems. This kind of game can be easily solved by mechine learning. However, the conditions in the game are variety and there are only a little level to let the meachine learn and the data is not universal, there must be a lot errors when using mechine learning.  When using some principles like Sudoku and Rubik's Cube to solve each level, efficiency will be greatly improved, and the turnover rate will be reduced. Meanwhile, from those methods, we can simulate the child's problem-solving thinking and to optimize the models.

\subsection{Analysis of Problem 1}

This problem requires us to develop a mathematical model to evaluate the difficulty of the game from each level. What we need to do is to analyze the limitation factor from the listed levels and we can analyze the quantity and the complexity of the steps to solve the problem. Whenever we take a step, the complexity of the game will drop with our steps. We can use entropy to represent this complexity. Whenever one step is taken, the corresponding entropy will decrease, and when the problem is solved, the entropy value will become 0.  The value of entropy is determined by the complexity of the game, so we can use entropy to assess the complexity of the game. The calculation of entropy is also determined by the constraints of the game, the type and quantity of animals, etc. Therefore, we can using computer to establish a kind of model about entropy to solve problem 1.

\subsection{Analysis of Problem 2}
This problem requires to solve the problems in the game by a model and only using the limitation factors that were listed in the game. To pass the game with the model, we can simulate Sudoku's problem solving process for they have similarities. When solving the Sudoku's problem, we always use back tracking method which is a kind of search algorithm.  The back tracking method can easily search the conditions and the location of each animal. It is like the algorithm of depth-first-search, which search step-by-step,and if there is a step that does not satisfy the conditions, abandoning this node and select another node to searching again. When abandoning a node, just like the tree  reduce one branch and the chances of finding solution will be greatly improved. And when we using this algorithm, the effectiveness to pass the game will  be improved. To know if the solution for this level is unique, what we need to do is regard each possible location as head node and using the algorithm of back tracking method to searching each possible solution.

\subsection{Analysis of Problem 3}
Designing games are based on good algorithms and great models which requires developers to have a good algorithm base and modeling capabilities. However, we have already well prepared for desigining games like "Wonder Island", because we have established models and given optimized algorithms in problem 1 and 2, and we can use those models and algorithms to design a game like "Wonder Island" which can produce levels automatically and the difficulty of each level will increase by the level-and-level. The game what we designed will  increase difficulty when the levels rising and it will  improve the ability to develop children's intelligence.

